\documentclass{llncs}

\usepackage{amsmath}
\usepackage{graphicx}
\usepackage{algorithm}
\usepackage{algpseudocode}
\usepackage{color}
\usepackage{amssymb}


\title{Fast DEM collision checks on multicore nodes}

\titlerunning{Fast collision checks on multicores}

\author{
%  \inst{1}
  Konstantinos Krestenitis
  \and
  Tobias Weinzierl
%  \inst{1}
\thanks{
This work has been sponsored by EPSRC (Engineering and Physical Sciences
Research Council) and EDF Energy as part of an ICASE studentship  (award ref 1429338). 
It made use of the facilities of the Hamilton HPC Service of Durham
University.
All software is freely available from \url{https://github.com/KonstantinosKr/delta}
(pronounced/written $\Delta $).
}
}

\institute{
  School of Engineering and Computing Sciences\\
  Durham University\\
  Great Britain \\
  \email{\{konstantinos.krestenitis,tobias.weinzierl\}@durham.ac.uk}
}

\begin{document}
\maketitle

\begin{abstract}
Many particle simulations today rely on spherical or analytical particle shape
descriptions.
They find non-spherical, triangulated particle models computationally
infeasible due to expensive collision detections.
We propose a hybrid collision detection algorithm based upon an iterative
solve of a minimisation problem that automatically falls back to a brute-force
comparison-based algorithm variant if the problem is ill-posed.
Such a hybrid can exploit the vector facilities of modern chips and it is
well-prepared for the arising manycore era.
Our approach pushes the boundary where non-analytical particle shapes
and the aligning of more accurate first principle physics become manageable.
\end{abstract}
 
\section{Introduction}
%
% Context of research; why is it important
% What makes these models reliable?
%
Discrete Element Methods (DEM) are a popular technique to model granular flow,
the break-up of brittle material, ice sheets, and many other phenomena.
They describe the medium of interest as a set of rigid bodies that interact with
each other through collisions and contact points.
The expressiveness of such a simulation is determined on the one hand by the
accuracy of the physical interaction model.
On the other hand, it is determined by the accuracy of scale: the more
rigid bodies (particles) can be simulated the more accurate the outcome.


%
% State-of-the-art role of physics
%

%\marginpar{@Konstantinos: Some hints
% Non-spherical particles reduce efficiency
%significantly [78,104] in book of Samiei 
%- Analytical shape functions besides spheres [41,56,61,118] in Semiei
%- Generic shape composition with spheres [75] und Samiei selber
%}
Many DEM codes restrict themselves to analytical shape models:
Their particles are described by some analytical function; most of the time spheres.
Furthermore, they stick to explicit time integrators (cmp. \cite{Boac2014} and
references therein).
Whenever particles are close to each other, i.e.~their distance underruns a
given threshold, they are assumed to be in contact. An interaction model then
realises two types of physics.
On the one hand, it mitigates the real-world impact of collision,
friction, and so forth.
On the other hand, it mitigates the fact that real particles are not
spherical/analytical \cite{Johnson2015}.


%
% What are shortcomings of today's solutions?
%
While the distributed memory parallelisation of DEM codes through classic
domain decomposition is well understood and the codes scale
(see \cite{Iglberger2010} as an example), most codes refrain from modelling
particles as irregularly shaped objects and, thus, from eliminating the second role of the interaction model, as
they already spend a majority of their compute time in collision detection.
Iglberger et al.~\cite{Iglberger2010} report 31--34\% within a multiphysics
setting, while Li \cite{Li1998} for example reports even 90\%.
Detection becomes significantly more complicated once we switch from
sphere-to-sphere or ellipsoid-to-ellipsoid checks to the comparison of billions
of triangles if the particles are represented by meshes---notably if no
constraints on convectivity are made and if explicit time stepping stops us
from modelling complex particle shapes as compound of simpler convex shapes
subject to a non-decomposable constraint.
Injecting meshes particles into DEM is a single node challenge.


%
% Our contribution plus structure of paper. Usually I prefer to separate them,
% but we are short of pages here.
%
We introduce a triangle-based collision detection scheme for DEM that
supports particles of arbitrary triangle count, configuration and size.
Geometric comparisons suffer from poor SIMDability if realised
straightforwardly as they involve many case distinctions.
We recast the geometric checks into a minimisation that falls back to
classic geometric checks as emergency solver.
This way, we obtain a collision detection algorithm that is both robust and can
exploit wide vector registers (Section \ref{section:penalty}).
Furthermore, it can be parallelised on multiple cores either by deploying the
triangles among multiple cores or by handling sets of triangles (batches)
concurrently (Section \ref{section:intra-particle}).
Some numerical results in Section \ref{section:results} highlight the potential
of our approach on multicore nodes before a brief summary and an outlook
detail the impact on future DEM codes.
\Chapter{Model}


\section{The particle and collision model}
\label{section:model}

%
% Definition of a particle 
% overall algorithmic sketch
% 
We study media composed of particles of arbitrary size.
Each particle $p_i \in \mathbb{P}$ is described  by a set of
triangles $\mathbb{T}_i$.
We do not impose any constraints on the triangle layout such as convexity.
Our algorithms of interest consist of an explicit time stepping loop with a time
step size $\Delta t$.
Per time step, it runs over all particle pairs and identifies where any
particle pair collides with each other: we determine the contact points.
Per contact point, we determine the arising forces on the involved particles.
Once all forces for all particles are accumulated, we update the particle velocities
and positions and continue.
If the contact point detection identifies that particles are too close to each
other it halves $\Delta t$.
If no contact points are identified at all, it increases $\Delta t$ by 10\%.

\begin{figure}[htb]
  \begin{center}
    \includegraphics[width=0.45\textwidth]{CP014_fig1}
  \end{center}
  \vspace{-0.6cm}
  \caption{
    Three particles with their $\epsilon $ environment. The particles do
    not penetrate each other, but two particles plus their $\epsilon $
    environment penetrate and create one contact point (diamond point) with a normal $n$. 
  }
  \label{figure:minkowski}
  \vspace{-0.4cm}
\end{figure}


%
% \epsilon environment
%
%
Our contact model is based upon an $\epsilon$ environment around each
particle and a weak compressibility model for this $\epsilon$-area:
Two particles are in contact as soon as they are closer than $2\epsilon$.
Mirroring Minkowski sums, we may interpret each particle to be enlarged by a
soft layer of width $\epsilon$ (Fig.~\ref{figure:minkowski}).
Particles are in contact with each other as soon as these soft areas penetrate.
If two particles are in contact, the contact point is the point that is closest
to the particles' surfaces.
We do not support contact areas at the moment but multiple contact points per
particle pair may exist.
Each contact point is associated one outer normal vector $n$ per involved
particle.
Though our particles themselves are rigid, we call $\epsilon-|n|$ between a contact point and the real particle surface the penetration depth.


%
% Spring dashpot
% Too close
%

Our force computation equals pseudo-elastic damping as it is used in
geometry overlapping methods \cite{Boac2014}.
We rely on the spring-dashpot DEM force model \cite{Cundall1979}
which yields per particle pair $p_i, p_j$ forces 

\begin{align} 
  f_{\bot}(p_i,p_j) &=
  \left\{
   \begin{array}{ll}
    S \cdot (\epsilon - |n_{ij}|) +2D \cdot \left(\sqrt{\frac{1.0}{\frac{1.0}{m_i} + \frac{1.0}{m_j}}}\right) \cdot (v_{ij},\frac{n_{ij}}{|n_{ij}|}) 
    &
    \mbox{if}\ (v_{i},v_{j})\leq 0 \\
    0 & \mbox{otherwise}
   \end{array}
  \right.,
  \nonumber
  \\
  f_{\parallel}(p_i,p_j) &= r \times f_{\top}(p_i,p_j)
    \label{equation:forces}
\end{align}



\noindent
acting on $p_i$. 
The forces on $p_j$ result from parameter permutation.
$(.,.)$ denotes the Euclidean scalar product, 
$D$ is a damping, $S$ the spring coefficient.
$v_{ij}$ is the relative collision velocity $v_j-v_i$.
$m$ denotes the mass of $p_i$ or $p_j$ respectively, $n_{ij}$ is the contact
normal pointing from the contact point in-between the particles onto the surface of particle $j$, i.e.~from $i$ to $j$.

The orthogonal force $f_{\bot}$ models solely repulsive forces, i.e.~forces
arise if and only if two particles continue to approach each other. 
The tangential force injects friction into the system.
Obviously, system (\ref{equation:forces}) is stiff and cannot avoid penetration
of the real particles without their halo environment.
We thus rely on small time step sizes $\Delta t$ and reduce $\Delta t$ as soon
as $|n| \leq 0.2 \cdot \epsilon $ for any contact point normal in the system.
$f_{\parallel}(p_i,p_j)$ is the torque force, where $r$ is the lever arm of $p_i$'s
centre of mass to the contact point.

%
% Complexity
% Linked-cell approach
% Brute force approach
% Constant size
%
The plain algorithm is in $\mathcal{O}(| \mathbb{P} |^2 \cdot
\mathbb{T}_{max}^2 )$ with $\mathbb{T}_{max} = max _i | \mathbb{T}_i | $.
We rely on a multiscale linked-cell approach based upon adaptive Cartesian
meshes as it is used in molecular dynamics codes
\cite{Griebel.Knapek.Zumbusch:2007}:
The computational domain is split into cubes that are at least as large as
the biggest particle in the system and the cubes host the particles, i.e.~hold
links to the particles.
The realisation stems from \cite{Weinzierl2016}.
Particles can be in contact if and only if they are hosted by the same or
neighbouring cells.
This reduces the first quadratic term to a linear one as
rigid particles cannot cluster arbitrarily dense.
The second quadratic term results from the fact that we have to compare, for two
particles $p_i$ and $p_j$, each triangle from particle $p_i$ with each triangle
from $p_j$.
Each pair of triangles requires fifteen checks: point-to-face
($2 \cdot 3=6$) and edge-to-edge ($3^2=9$).
These comparisons are based upon a barycentric coordinate transform and yield
a sequence of computations followed by if statements filtering out
inadmissible solutions.
The 15 distance computations then are reduced subject to the minimum function.
Vectorisation of this approach labelled {\em brute force} suffers from branching
and low arithmetic intensity.

\section{Results}
\label{section:intra-particle}

Our multithreaded collision detection code runs a classic data decomposition
scheme on the triangles:
While the first triangle $T_i$ of $p_i$ is compared to the first triangle of $T_j$, we can simultaneously compare the second triangle. 
The concurrency scales in the number of triangles per particle.
Synchronisation points, i.e.~critical sections, are solely the insertion of
contact points into the result set.

\input{conclusion}

\bibliographystyle{siam}
\bibliography{paper} 

\end{document}
