\Chapter{Model}


\section{The particle and collision model}
\label{section:model}

%
% Definition of a particle 
% overall algorithmic sketch
% 
We study media composed of particles of arbitrary size.
Each particle $p_i \in \mathbb{P}$ is described  by a set of
triangles $\mathbb{T}_i$.
We do not impose any constraints on the triangle layout such as convexity.
Our algorithms of interest consist of an explicit time stepping loop with a time
step size $\Delta t$.
Per time step, it runs over all particle pairs and identifies where any
particle pair collides with each other: we determine the contact points.
Per contact point, we determine the arising forces on the involved particles.
Once all forces for all particles are accumulated, we update the particle velocities
and positions and continue.
If the contact point detection identifies that particles are too close to each
other it halves $\Delta t$.
If no contact points are identified at all, it increases $\Delta t$ by 10\%.

\begin{figure}[htb]
  \begin{center}
    \includegraphics[width=0.45\textwidth]{CP014_fig1}
  \end{center}
  \vspace{-0.6cm}
  \caption{
    Three particles with their $\epsilon $ environment. The particles do
    not penetrate each other, but two particles plus their $\epsilon $
    environment penetrate and create one contact point (diamond point) with a normal $n$. 
  }
  \label{figure:minkowski}
  \vspace{-0.4cm}
\end{figure}


%
% \epsilon environment
%
%
Our contact model is based upon an $\epsilon$ environment around each
particle and a weak compressibility model for this $\epsilon$-area:
Two particles are in contact as soon as they are closer than $2\epsilon$.
Mirroring Minkowski sums, we may interpret each particle to be enlarged by a
soft layer of width $\epsilon$ (Fig.~\ref{figure:minkowski}).
Particles are in contact with each other as soon as these soft areas penetrate.
If two particles are in contact, the contact point is the point that is closest
to the particles' surfaces.
We do not support contact areas at the moment but multiple contact points per
particle pair may exist.
Each contact point is associated one outer normal vector $n$ per involved
particle.
Though our particles themselves are rigid, we call $\epsilon-|n|$ between a contact point and the real particle surface the penetration depth.


%
% Spring dashpot
% Too close
%

Our force computation equals pseudo-elastic damping as it is used in
geometry overlapping methods \cite{Boac2014}.
We rely on the spring-dashpot DEM force model \cite{Cundall1979}
which yields per particle pair $p_i, p_j$ forces 

\begin{align} 
  f_{\bot}(p_i,p_j) &=
  \left\{
   \begin{array}{ll}
    S \cdot (\epsilon - |n_{ij}|) +2D \cdot \left(\sqrt{\frac{1.0}{\frac{1.0}{m_i} + \frac{1.0}{m_j}}}\right) \cdot (v_{ij},\frac{n_{ij}}{|n_{ij}|}) 
    &
    \mbox{if}\ (v_{i},v_{j})\leq 0 \\
    0 & \mbox{otherwise}
   \end{array}
  \right.,
  \nonumber
  \\
  f_{\parallel}(p_i,p_j) &= r \times f_{\top}(p_i,p_j)
    \label{equation:forces}
\end{align}



\noindent
acting on $p_i$. 
The forces on $p_j$ result from parameter permutation.
$(.,.)$ denotes the Euclidean scalar product, 
$D$ is a damping, $S$ the spring coefficient.
$v_{ij}$ is the relative collision velocity $v_j-v_i$.
$m$ denotes the mass of $p_i$ or $p_j$ respectively, $n_{ij}$ is the contact
normal pointing from the contact point in-between the particles onto the surface of particle $j$, i.e.~from $i$ to $j$.

The orthogonal force $f_{\bot}$ models solely repulsive forces, i.e.~forces
arise if and only if two particles continue to approach each other. 
The tangential force injects friction into the system.
Obviously, system (\ref{equation:forces}) is stiff and cannot avoid penetration
of the real particles without their halo environment.
We thus rely on small time step sizes $\Delta t$ and reduce $\Delta t$ as soon
as $|n| \leq 0.2 \cdot \epsilon $ for any contact point normal in the system.
$f_{\parallel}(p_i,p_j)$ is the torque force, where $r$ is the lever arm of $p_i$'s
centre of mass to the contact point.

%
% Complexity
% Linked-cell approach
% Brute force approach
% Constant size
%
The plain algorithm is in $\mathcal{O}(| \mathbb{P} |^2 \cdot
\mathbb{T}_{max}^2 )$ with $\mathbb{T}_{max} = max _i | \mathbb{T}_i | $.
We rely on a multiscale linked-cell approach based upon adaptive Cartesian
meshes as it is used in molecular dynamics codes
\cite{Griebel.Knapek.Zumbusch:2007}:
The computational domain is split into cubes that are at least as large as
the biggest particle in the system and the cubes host the particles, i.e.~hold
links to the particles.
The realisation stems from \cite{Weinzierl2016}.
Particles can be in contact if and only if they are hosted by the same or
neighbouring cells.
This reduces the first quadratic term to a linear one as
rigid particles cannot cluster arbitrarily dense.
The second quadratic term results from the fact that we have to compare, for two
particles $p_i$ and $p_j$, each triangle from particle $p_i$ with each triangle
from $p_j$.
Each pair of triangles requires fifteen checks: point-to-face
($2 \cdot 3=6$) and edge-to-edge ($3^2=9$).
These comparisons are based upon a barycentric coordinate transform and yield
a sequence of computations followed by if statements filtering out
inadmissible solutions.
The 15 distance computations then are reduced subject to the minimum function.
Vectorisation of this approach labelled {\em brute force} suffers from branching
and low arithmetic intensity.
